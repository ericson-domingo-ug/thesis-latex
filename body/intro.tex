\section{Background of the Study}
	The urban heat island refers to the phenomenon where urban areas have higher temperatures compared to rural areas.
	
\section{Significance of the Study}
	The United Nations Sustainability Goals blah blah blah.
	
	This study is also an opportunity to try the newest version of RegCM, developed by the Abdus Salam International Centre for Theoretical Physics.
	RegCM is a free, limited area model for long-term regional climate simulation.
	The newest version, RegCM5, was released in 2023 and will be used in this study. The previous version, RegCM4, was released in 2012.
	

\section{Objectives}
	The general objective of the study is to use RegCM5 to forecast the effect of the urban heat island in Metro Manila on air temperature in the future.
	The specific objectives are:
	\begin{enumerate}
		\item determine the performance of RegCM5 in simulating near-surface air temperature in Metro Manila;
		\item to analyze the trends of near-surface air temperature and urban heat island intensity in Metro Manila; and
		\item to forecast the near-surface air temperature and urban heat island intensity of Metro Manila to the year 2040.
		
	\end{enumerate}
	

\section{Scope and Delimitations}
	This study will focus on Metro Manila.
	This study will study two time periods: 
		the present represented by the years 2004 to 2023, 
		and the future represented by the years 2021 to 2040.
	
	The near-surface air temperature of Metro Manila will be analyzed.
	
	The RegCM software will be used.
	
	Anthropogenic heat will be ignored in this study. 