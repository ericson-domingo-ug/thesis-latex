\section{Background of the Study}
	The urbanization of a settlement greatly changes its environment.
	In an urban city, the artificial surfaces, the physical structures, and man-made processes contribute to the modification of the city's climate.
	One effect of urbanization can be seen in the urban heat island.
	The urban heat island refers to the phenomenon where urban areas have higher temperatures compared to rural areas.
	
\section{Significance of the Study}
	One of these goals of the United Nations' Sustainable Development Goals (SDG) is SDG 11: Sustainable Cities and Communities.
	The goal is to ``make cities and human settlements inclusive, safe, resilient and sustainable'' (\cite{UN2015}).
	In line with SDG 11 is the mitigation of the urban heat island phenomenon, as it is a threat to the health of the citizens in the city.
	High temperatures can cause discomfort (\cite{Bhati2018}), worsen ambient air quality, and increase morbidity and mortality (\cite {Khan2021}).
	The effects of the urban heat island are also intensified by climate change.
	Thus, it is important to study this phenomenon in order to understand it better and to learn how to mitigate it.	

\section{Objectives}
	The general objective of the study is to use RegCM5 to forecast the effect of the urban heat island in Metro Manila on air temperature in the future.
	The specific objectives are:
	\begin{enumerate}
		\item determine the performance of RegCM5 in simulating near-surface air temperature in Metro Manila;
		\item to analyze the trends of near-surface air temperature and urban heat island intensity in Metro Manila; and
		\item to forecast the near-surface air temperature and urban heat island intensity of Metro Manila to the year 2040.
		
	\end{enumerate}
	

\section{Scope and Delimitations}
	This study will focus on Metro Manila.
	This study will study two time periods: 
		the present represented by the years 2004 to 2023, 
		and the future represented by the years 2021 to 2040.
	
	The near-surface air temperature of Metro Manila will be analyzed.
	
	The RegCM software will be used for the simulations.
	RegCM is a free, limited area model for long-term regional climate simulation.
	The model is developed by the Abdus Salam International Centre for Theoretical Physics.
	The newest version, RegCM5, was released in 2023 and will be used in this study.
	This latest version is described in detail by \textcite{Giorgi2023}. 
	The previous version, RegCM4, was released in 2012 (\textcite{Giorgi2012}).
	
	Anthropogenic heat will be ignored in this study. 