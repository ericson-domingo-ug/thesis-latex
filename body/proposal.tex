\section{Preliminary Simulations}
	In order to test if RegCM5 is working on the laboratory's workstation,
		a preliminary simulation was conducted.
	The domain of the simulation is Luzon, centered on Manila, as shown in Figure \ref{fig:proposal-domain}.
	The domain has an area of $\num{130}$ by $\num{124}$ grid cells, with a $\qty{3}{km}$ resolution.
	Data for initial and boundary condition were dated from March to September 1990,
		and the simulation itself is from March to May 1990.
	The run used 24 cores and took 14 hours to complete.
		
	\begin{figure}
		\centering
		\includegraphics{proposal-domain}
		\caption{
			Surface model elevation of the domain used in the preliminary simulation.
			Domain is centered on Manila (\ang{14;35} N, \ang{121} E),
				with a grid of $\num{130}$ by $\num{124}$ cells
				and a resolution of $\qty{3}{km}$.
			Units of the elevation are in meters.
		}
		\label{fig:proposal-domain}
	\end{figure}
	
	Figure \ref{fig:proposal-manila-results} shows a comparison of the results of the simulation and the observed temperature in Manila for the month of March, 1990.
	Visually, the simulation somewhat matches the observed data, though there are large differences.
	Figures \ref{fig:proposal-naia-results} and \ref{fig:proposal-sciencegarden-results} show the same comparison as with Figure \ref{fig:proposal-manila-results}, 
		but for the Ninoy Aquino International Airport in Pasay City and the Science Garden in Quezon City, respectively.
	Visually, both cities exhibit higher simulated air temperature of around $+ \qty{2}{\degreeCelsius}$ compared to the observed data. 
	
	
	\begin{figure}
		\centering
		\begin{subfigure}{\textwidth}
			\includegraphics[width=\textwidth]{proposal-manila-both}
		\end{subfigure}
		\begin{subfigure}{\textwidth}
			\includegraphics[width=\textwidth]{proposal-manila-difference}
		\end{subfigure}
		\caption{
			Graphs of the simulated and actual near-surface air temperature of Manila (\ang{14.58} N, \ang{120.98} E).
		}
		\label{fig:proposal-manila-results}
	\end{figure}
	
	\begin{figure}
		\centering
		\begin{subfigure}{\textwidth}
			\includegraphics[width=\textwidth]{proposal-naia-both}
		\end{subfigure}
		\begin{subfigure}{\textwidth}
			\includegraphics[width=\textwidth]{proposal-naia-difference}
		\end{subfigure}
		\caption{
			Graphs of the simulated and actual near-surface air temperature of the Ninoy Aquino International Airport, Pasay (\ang{14.51} N, \ang{120.02} E).
		}
		\label{fig:proposal-naia-results}
	\end{figure}

	\begin{figure}
		\centering
		\begin{subfigure}{\textwidth}
			\includegraphics[width=\textwidth]{proposal-sciencegarden-both}
		\end{subfigure}
		\begin{subfigure}{\textwidth}
			\includegraphics[width=\textwidth]{proposal-sciencegarden-difference}
		\end{subfigure}
		\caption{
			Graphs of the simulated and actual near-surface air temperature of the Science Garden, Quezon City (\ang{14.65} N, \ang{120.05} E).
		}
		\label{fig:proposal-sciencegarden-results}
	\end{figure}

\section{Expected Results}
	The results will be divided into two sections:
		one for the simulation of the present,
		and one for the simulation of the future.
	For both simulations,
	various descriptive statistics will be computed, such as the mean air temperature and the maximum and minimum air temperature for each month.
	A map of the average air temperature of Metro Manila will be made, similar to Figure \ref{fig:rrl-almadronesreyes2022-mm}.
	
	For the simulation of the present,
	evaluation of the simulation shall be performed and its results will be tabulated.
	A sample table of the statistical performance of the simulation is given in Table \ref{tab:sample-statistical-performance}.
	A line graph of the air temperatures comparing the simulated and actual values will be made too, similar to Figures \ref{fig:proposal-manila-results} to \ref{fig:proposal-sciencegarden-results}.
	The simulation of the present time period will take approximately a month to simulate.

	\begin{table}[]
		\caption{
			Sample table of the statistical performance of the simulation.
			Data taken from \textcite{Bilang2022}.
		}
		\label{tab:sample-statistical-performance}
		\centering
		\begin{tabular}{lrrrr}
			\hline \hline
			Location                    & MB (\unit{\degreeCelsius})    & MAE (\unit{\degreeCelsius}) & RMSE (\unit{\degreeCelsius}) & IOA \\
			\hline
			NAIA, Pasay City            & -0.91 & 1.15 & 1.40 & 0.90 \\
			Manila City                 & -0.09 & 0.68 & 0.98 & 0.94 \\
			Science Garden, Quezon City & 1.56  & 1.95 & 2.46 & 0.79 \\
			\hline
		\end{tabular}
	\end{table}

	For the simulation of the future,
	a line graph of the simulated air temperature will be graphed, similar to Figures \ref{fig:proposal-manila-results} to \ref{fig:proposal-sciencegarden-results}, but without the observed data.
	A linear regression will be applied to see the rate at which the air temperature changes.
	The simulation of the future will also take approximately a month to simulate.
	
 	After the results will be a discussion of the findings.
	There will be a discussion of what the results imply for the future of the cities studied.
	There will also be a discussion of the limitations of the simulation model how that affects the results.
