\section{Radiative Transfer}

	The sun heats up the atmosphere and surface of Earth through radiative transfer, the transfer of energy via electromagnetic radiation.
	When radiation interacts with matter, it may either be absorbed, emitted, reflected, scattered, or simply transmitted.
	The outcome depends on the wavelength of the radiation as well as the properties of the material interacting with the radiation.

	\subsection{Reflection and Scattering}
	%
	%As solar radiation travels through the atmosphere,

	\subsection{Absorption and Emission}
	

\section{Urban Heat Island}

	\subsection{Categories of Urban Heat Island}
	
	There are two types of urban heat island based on how they are formed and how high they reach:
	surface urban heat islands and atmospheric urban heat islands.

	
	Surface urban heat islands refer to the warmer surface of urbanized areas compared to the temperature of rural surfaces, and is primarily measured by satellite thermal remote sensing data (\cite{Zhou2018}). 
	How hot a surface can reach depends on its properties.
	Dry and exposed surfaces such as roofs and pavements can become significantly hotter than the air, while shaded and moist surfaces remain as hot as the air (\cite{Khan2021}). 
	
	Atmospheric urban heat islands refer to the warmer air temperature of urbanized areas compared to the air temperature of rural areas.
	This category of urban heat island is subdivided into two more classifications:
	the canopy layer and the boundary layer (\cite{Zhou2018}).
	The canopy layer starts from the ground up to the treetops and rooftops, 
	while the boundary starts from the treetops and rooftops and extends up until point where the urban area does not affect the atmosphere (\cite{Khan2021}).
	Canopy temperatures are typically measured by sensors from meteorological stations or vehicles,
	while boundary temperatures are measured by sensors from specialized platforms such as tall towers or aircrafts (\cite{Zhou2018}).

	\subsection{Factors Causing Urban Heat Islands}
	
	Different climatic and nonclimatic factors are responsible for causing urban heat islands.
	\citeauthor{Bridgman1995} (\citeyear{Bridgman1995}, as cited in \cite{Khan2021}) lists five major ways in which urbanization can influence a city's climate:
	\begin{itemize}
		\item by replacing natural surfaces with asphalts, concrete, and glasses;
		\item by replacing natural shape with blocky, angular, towering structures;
		\item by releasing anthropogenic heat into the urban atmosphere;
		\item by routing flow of surface runoff and preventing infiltration; and
		\item by emitting pollutants into the urban atmosphere.
	\end{itemize}