\section{Rural Energy Balance}
	The heating of the Earth is caused by the exchange of energy between the Sun, the Earth's atmosphere, and the Earth's surface.
	The solar rays incident on the Earth, combined with the Earth's rotation, creates a diurnal energy balance.
	These energy exchanges can be associated with an energy flux density, with units $\si{W.m^{-2}}$.
	
	Let $Q_{S\downarrow}$ be the magnitude of the flux of shortwave radiation that reaches the surface, which is integrated over all wavelengths in and near	the visible spectrum.
	The surface reflects some of the sunlight back upward, of magnitude $Q_{S\uparrow}$.
	Also, the atmosphere emits longwave radiation, some of which
	$Q_{L\downarrow}$ reaches the Earth’s surface.
	The Earth’s surface emits longwave radiation upward, with flux magnitude $Q_{L\uparrow}$. 
	The sum of the inputs to the surface minus the outputs yields the net radiation flux $Q^*$ absorbed at the
	surface
	\begin{equation}
		Q^* = Q_{S\downarrow} - Q_{S\uparrow} + Q_{L\downarrow} - Q_{L\uparrow}.
	\end{equation}
	%TODO: rewrite this paragraph
	
	The surplus of energy provided by $Q^*$ is shared by the soil and the air.
	Heat is dissipated via conduction to the soil with flux $Q_G$,
		and via convection of sensible and latent heat with fluxes $Q_H$ and $Q_E$ respectively. 
	Thus,
	\begin{equation}
		Q^* = Q_G + Q_H + Q_E
	\end{equation}

\section{Urban Energy Balance}
	The urban heat island effect is the result of the differences between energy balance of the rural environment and the urban environment.
	The urban environment modifies the energy balance in multiple ways:
	\begin{enumerate}
		\item \textbf{Canyon geometry of buildings.}
		Dense tall buildings create a canyon geometry.
		This causes radiation to get trapped in between the vertical surfaces through multiple reflections.
		This also causes a reduction of wind speed, which trap warm air.
		
		\item B
		
		\item C
		
		\item D
		
		\item \textbf{Anthropogenic heat.}
		There is a greater release of heat from man-made activities such as from
			the combustion of fuels in vehicles,
			industrial processes, and
			air-conditioning in rooms.
		Humans also release heat and moisture from metabolism, but is not as significant compared to the activities mentioned.
	\end{enumerate}

%\section{Radiative Transfer}
%
%	The sun heats up the atmosphere and surface of Earth through radiative transfer, the transfer of energy via electromagnetic radiation.
%	When radiation interacts with matter, it may either be absorbed, emitted, reflected, scattered, or simply transmitted.
%	The outcome depends on the wavelength of the radiation as well as the properties of the material interacting with the radiation.
%
%	\subsection{Reflection, Scattering, and Albedo}
%	%
%	%As solar radiation travels through the atmosphere,
%	%\blindtext
%	Energy may be returned to space through reflection or scattering.
%	
%	Albedo is defined as the ``ratio of the amount of solar radiation reflected by a surface to the amount received by it'' (\cite{Stewart2012}).
%	
%
%	\subsection{Absorption and Emission}
%	
%	\blindtext
%
%\section{Urban Heat Island}
%
%	\subsection{Categories of Urban Heat Island}
%	
%	There are two types of urban heat island based on how they are formed and how high they reach:
%	surface urban heat islands and atmospheric urban heat islands.
%
%	Surface urban heat islands refer to the warmer surface of urbanized areas compared to the temperature of rural surfaces, and is primarily measured by satellite thermal remote sensing data (\cite{Zhou2018}). 
%	How hot a surface can reach depends on its properties.
%	Dry and exposed surfaces such as roofs and pavements can become significantly hotter than the air, while shaded and moist surfaces remain as hot as the air (\cite{Khan2021}). 
%	
%	Atmospheric urban heat islands refer to the warmer air temperature of urbanized areas compared to the air temperature of rural areas.
%	This category of urban heat island is subdivided into two more classifications:
%	the canopy layer and the boundary layer (\cite{Zhou2018}).
%	The canopy layer starts from the ground up to the treetops and rooftops, 
%	while the boundary starts from the treetops and rooftops and extends up until point where the urban area does not affect the atmosphere (\cite{Khan2021}).
%	Canopy temperatures are typically measured by sensors from meteorological stations or vehicles,
%	while boundary temperatures are measured by sensors from specialized platforms such as tall towers or aircrafts (\cite{Zhou2018}).
%
%	\subsection{Factors Causing Urban Heat Islands}
%	
%	Different climatic and nonclimatic factors are responsible for causing urban heat islands.
%	\citeauthor{Bridgman1995} (\citeyear{Bridgman1995}, as cited in \cite{Khan2021}) lists five major ways in which urbanization can influence a city's climate:
%	\begin{itemize}
%		\item by replacing natural surfaces with asphalts, concrete, and glasses;
%		\item by replacing natural shape with blocky, angular, towering structures;
%		\item by releasing anthropogenic heat into the urban atmosphere;
%		\item by routing flow of surface runoff and preventing infiltration; and
%		\item by emitting pollutants into the urban atmosphere.
%	\end{itemize}