\section{Overview on Urban Heat}
	\blindtext
	%TODO: add information on anthropogenic heat
	
\section{Studies of Urban Heat in North America and Europe}
	\blindtext
	
\section{Studies of Urban Heat in South Asia and South East Asia}
	\textcite{Yuan2020} conducted a computational parametric study in order to evaluate the impact of urbanization in Singapore.
	Using Computational Fluid Dynamics, they examined how anthropogenic heat disperses given the city's urban morphology.
	The authors discovered that heat is hard to disperse if there is a lack of incoming wind.
	Based on the findings, the authors created a practical modeling tool in order to help with urban planning. 
	The Geographic Information System-based tool estimates how much impact anthropogenic heat has on the city's ambient temperature.
	
	\textcite{Gao2019} explored how effective different strategies are in mitigating the surface urban heat island in Wuhan China.
	Using the offline urbanized High-Resolution Land Data Assimilation System (u-HRLDAS), they showed that the best mitigation strategy for daytime urban heat islands is using green roofs and cool roofs.
	In their follow-up study (\cite{Gao2020}), the authors compared mitigation strategies in Wuhan, China to Xi'an, China.
	They found that all the mitigation strategies they considered were more effective for Xi'an than it was for Wuhan. 
	The authors explain that this is due to the differences in regional climate.