\section{Overview on Urban Heat}
	\blindtext
	%TODO: add information on anthropogenic heat
	%

\section{Summary of Studies on Urban Heat}	
	\subsection{North America and Europe}
		\blindtext
		
	\subsection{South Asia and South East Asia}
		\textcite{Yuan2020} conducted a computational parametric study in order to evaluate the impact of urbanization in Singapore.
		Using Computational Fluid Dynamics, they examined how anthropogenic heat disperses given the city's urban morphology.
		The authors discovered that heat is hard to disperse if there is a lack of incoming wind.
		Based on the findings, the authors created a practical modeling tool in order to help with urban planning. 
		The Geographic Information System-based tool estimates how much impact anthropogenic heat has on the city's ambient temperature.
		
		\textcite{Gao2019} explored how effective different strategies are in mitigating the surface urban heat island in Wuhan China.
		Using the offline urbanized High-Resolution Land Data Assimilation System (u-HRLDAS), they showed that the best mitigation strategy for daytime urban heat islands is using green roofs and cool roofs.
		In their follow-up study (\cite{Gao2020}), the authors compared mitigation strategies in Wuhan, China to Xi'an, China.
		They found that all the mitigation strategies they considered were more effective for Xi'an than it was for Wuhan. 
		The authors explain that this is due to the differences in regional climate.
		
		

	\subsection{The Philippines}
		The quick brown fox blah blah blah.

\section{Common Themes}
	Studies report a higher temperature due to urbanization, which is to be expected.
	Reported temperature changes go from 1 \degree C (\cite{Huszar2018a}), 
		up to 10 \degree C (\cite{Santamouris2020}).
		%up to 8 \degree C (\cite{Wang2019}).
	For the Philippines specifically, reported temperature changes due to the urban heat island effect go from 0.4 \degree C (\cite{Oliveros2019}),
		up to 6 \degree C (\cite{Purio2022}).
	
	There does not seem to be one common way of simulating urban heat.
	A handful of studies use the Weather Research \& Forecasting Model (WRF) with an urban canopy model.
	Other simulations include the Regional Climate Model (RegCM), and
		the urbanized high-resolution land data assimilation system (u-HRLDAS).
		
	
	
	
	
	
		
	%technique, results,
		%ano basehan na may urban heat problem?
		%paano sinusukat ang basehan na yun?
		%why select this location?
		%representative ba siya?
		%areas
			%materials used in buildings, its climate, what's in the area (mostly bahay? mostly trees? tubig?)
	%theory part: the physics!
		%equations
		%what simulation
	%method
		%analytical data used
	%resap2: theory. resap3: method (together with budget gant chart intro).
	%BFAR DENR DOST-SEI
	%urban heat isalands
	%yung magpappatagaL: YUNG mga runs!