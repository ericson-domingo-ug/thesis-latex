\section{Area of Study}
	This study will be conducted over Metro Manila, Philippines.
	Metro Manila, formally known as the National Capital Region, is the capital region of the Philippines.
	The region has a total area of 620 square kilometers (citation)
		and has a population of 13 million people (citation).
	It is composed of one municipality: Pateros, and sixteen cities:
		Caloocan,
		Las Pinas,
		Makati,
		Malabon,
		Mandaluyong,
		Manila,
		Marikina,
		Muntinlupa,
		Navotas,
		Paranaque,
		Pasay,
		Pasig,
		Quezon City,
		San Juan,
		Taguig, and
		Valenzuela.
		
\section{Simulation Model}
	This study will use the Regional Climate Model version 5 (RegCM5), which is described in detail by \textcite{Giorgi2023}.
	RegCM5 is developed by the Abdus Salam International Centre for Theoretical Physics.
	
	\begin{table}
		\label{tab:physics-schemes}
		\caption{Configuration of the RegCM5 physics schemes.}
		\centering
		\begin{tabular}{l l}
			\hline \hline
			Physics scheme & Configuration\\
			\hline
			Radiation scheme & NCAR CCM3 \\
			Land surface model & CLM4.5 \\
			Planetary boundary layer scheme & Based on \textcite{Holtslag1990}\\
			Cumulus convection scheme & Based on \textcite{Emanuel1991}\\
			\hline
		\end{tabular}		
	\end{table}