\section{Area of Study}
	This study will be conducted over Metro Manila, Philippines.
	Metro Manila, formally known as the National Capital Region, is the capital region of the Philippines.
	The region has a total area of 620 square kilometers (citation)
		and has a population of 13 million people (citation).
	It is composed of one municipality: Pateros, and sixteen cities:
		Caloocan,
		Las Pinas,
		Makati,
		Malabon,
		Mandaluyong,
		Manila,
		Marikina,
		Muntinlupa,
		Navotas,
		Paranaque,
		Pasay,
		Pasig,
		Quezon City,
		San Juan,
		Taguig, and
		Valenzuela.
		
\section{Simulation Model}
	This study will use the latest version of the Regional Climate Model, RegCM5, which is described in detail by \textcite{Giorgi2023}.
	RegCM5 is a limited area model for long-term regional climate simulation.
	It is developed by the Abdus Salam International Centre for Theoretical Physics.
	The previous version, RegCM4, was released in 2012 (\cite{Giorgi2012}).
	
	\begin{table}	
		\caption{Configuration of the RegCM5 physics schemes.}
		\label{tab:physics-schemes}
		\centering
		\begin{tabular}{p{2 in} p{2.75 in}}
			\hline \hline
			Physics scheme & Configuration\\
			\hline
			Atmospheric radiation & Radiation scheme from the Community Climate Model version 3 (\cite{Kiehl1996}) \\
			Land surface model & Community Land Model version 4.5 (\cite{Oleson2013})\\
			Planetary boundary layer & Based on \textcite{Holtslag1990}\\
			Cumulus convection & Based on \textcite{Emanuel1991}\\
			Resolvable scale precipitation & Subgrid explicit moisture scheme (SUBEX) (\cite{Pal2000})\\
			\hline
		\end{tabular}		
	\end{table}

	This study will conduct simulations over two 20-year periods.
	The first simulation will be conducted from 2004 to 2023 to study the present,
		while the second simulation will be conducted from 2021 to 2040 to study the future.
	Table \ref{tab:physics-schemes} shows the physics schemes to be used in the study. 
	
\section{Simulation Evaluation}
	To determine the performance of the simulation, its results will be compared to real data from the Integrated Surface Database (ISD).
	The ISD is maintained by the United States National Oceanic and Atmospheric Administration, and is readily available on their website 
		(https://www.ncei.noaa.gov/products/land-based-station/integrated-surface-database).
	
	Four performance statistics will be computed: 
		the mean bias (MB),
		the root mean square error (RMSE),
		the mean absolute error (MAE), and
		the index of agreement (IOA).
	These statistics are computed using equations N to N,
		where $y_i$ is the modeled value, $y_{i,\text{obs}}$ is the observed value, and $N$ is the number of data points:
	\begin{align}
		\text{MB} &= b \\
		\text{RMSE} &= d \\
		\text{MAE} &= b \\
		\text{IOA} &= d \\
		\bar{y} &= d \\
	\end{align}
	