\chapter*{Abstract}

The urbanization of a settlement can make its environment have higher temperatures compared to its nearby rural areas.
This phenomenon is known as the urban heat island effect.
One method of studying the urban heat island effect is through computer simulations, and in order to have accurate results, the performance of the computer simulation to be used needs to be evaluated.
This study aims to evaluate the performance of the latest version of the Regional Climate Model (RegCM5) developed by the International Center for Theoretical Physics, in simulating the near-surface air temperature in six highly-urbanized cities in Luzon, Philippines: Angeles, Baguio, Manila, Olongapo, Pasay, and Quezon.
Four simulations were run with two different datasets of initial conditions and boundary conditions (ICBC): EIN15 and CNRM-CM5, and two different horizontal resolutions: 16 km and 8 km. The results of the simulations were compared to weather station data from the Integrated Surface Dataset (ISD), maintained by the National Oceanic and Atmospheric Administration.
Four evaluation statistics were computed: the mean bias (MB), mean absolute error (MAE), and root mean square error (RMSE) show the closeness of the simulation to observed data, and the index of agreement (IOA) shows how error-free a model prediction is.
Results show among the four runs, the run using EIN15 as its ICBC with 8 km resolution performs the best, showing MB, MAE, and RMSE values below $\pm \qty{2.0}{\degreeCelsius}$, $\qty{2.0}{\degreeCelsius}$, and $\qty{3.5}{\degreeCelsius}$, respectively.
Furthermore, the run exhibits high IOA ($\geq \num{0.8}$) in all cities except Olongapo. The runs using CNRM-CM5 perform poorly in both resolutions, with most IOA values $< \num{0.8}$, which indicate that a finer resolution is needed in order for it to be more accurate.
%TODO: add implications

\vspace{3ex} \noindent {\itshape Key Words}---near-surface air temperature; RegCM5; urban heat islands; Luzon; Philippines.

%Keep your abstract short by giving the gist/nutshell of your \MakeTextLowercase{\documentType}.	 Use the following checklist questions to help you in crafting your abstract.	
%
%\begin{itemize}
%	\item[$\square$] Did you briefly state what you intend to do?  
%	\item[$\square$] Did you concisely discuss the problem statement?
%	\item[$\square$] Did you tersely mention the objectives in general terms? 
%	\item[$\square$] Did you succinctly describe the methodology for the target audience?
%	\item[$\square$] Did you strongly describe your significant results and your conclusions?
%\end{itemize}