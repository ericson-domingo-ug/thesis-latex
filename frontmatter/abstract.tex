\chapter*{Abstract}

The urbanization of a settlement can make its environment have higher temperatures compared to its nearby rural areas,
	a phenomenon known as the urban heat island effect.
This study aims to forecast the effect of the urban heat island on near-surface air temperature in select highly-urbanized cities of Luzon, Philippines: Angeles, Baguio, Manila, Olongapo, Pasay, and Quezon.
The simulation used is the Regional Climate Model (RegCM5), developed by the International Center for Theoretical Physics.
Model evaluation was first conducted in order to evaluate the accuracy of the model to simulate.
The results of the sensitivity runs were compared to weather station data from the Integrated Surface Dataset (ISD), maintained by the National Oceanic and Atmospheric Administration.
A hindcast from 2013-2018 and forecast from 2027-2031 was conducted using best settings from model evaluation, and a SARIMA model was applied to hindcast data in order to see trends in data.
Results show among the sensitivity runs, the run using EIN15 as its ICBC with 8 km resolution performs the best, showing mean bias, mean absolute error, and root mean square error values below $\pm \qty{2.0}{\degreeCelsius}$, $\qty{2.0}{\degreeCelsius}$, and $\qty{3.5}{\degreeCelsius}$, respectively.
Furthermore, the run exhibits high IOA ($\geq \num{0.8}$) in all cities except Olongapo.
Simulation shows that more urbanized cities have a lower difference between daytime and nighttime temperature compared to less urbanized ones.
Maximum urban heat island intensity can reach 
$\qtyrange{4.1}{7.7}{\degreeCelsius}$ from 2013 to 2018, and
$\qtyrange{4.2}{8.2}{\degreeCelsius}$ from 2027-2031 for the studied cities.
SARIMA forecasts to have the mean near-surface air temperature of cities increase by $\qtyrange{0.9}{1.0}{\degreeCelsius}$ by 2030.
The study overall shows that RegCM5 can be accurate in simulating urban heat, but can be improved with finer resolution and coupling with other models, and that 
the urban heat island effect will be greater by 2030.

\vspace{3ex} \noindent {\itshape Key Words}---near-surface air temperature; RegCM5; urban heat islands; Luzon; Philippines.

%Keep your abstract short by giving the gist/nutshell of your \MakeTextLowercase{\documentType}.	 Use the following checklist questions to help you in crafting your abstract.	
%
%\begin{itemize}
%	\item[$\square$] Did you briefly state what you intend to do?  
%	\item[$\square$] Did you concisely discuss the problem statement?
%	\item[$\square$] Did you tersely mention the objectives in general terms? 
%	\item[$\square$] Did you succinctly describe the methodology for the target audience?
%	\item[$\square$] Did you strongly describe your significant results and your conclusions?
%\end{itemize}